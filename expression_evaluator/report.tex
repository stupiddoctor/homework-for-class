\documentclass[UTF8]{ctexart}
\usepackage{geometry, CJKutf8}
\geometry{margin=1.5cm, vmargin={0pt,1cm}}
\setlength{\topmargin}{-1cm}
\setlength{\paperheight}{29.7cm}
\setlength{\textheight}{25.3cm}

% useful packages.
\usepackage{amsfonts}
\usepackage{amsmath}
\usepackage{amssymb}
\usepackage{amsthm}
\usepackage{enumerate}
\usepackage{graphicx}
\usepackage{multicol}
\usepackage{fancyhdr}
\usepackage{layout}
\usepackage{listings}
\usepackage{float, caption}

\lstset{
    basicstyle=\ttfamily, basewidth=0.5em
}

% some common command
\newcommand{\dif}{\mathrm{d}}
\newcommand{\avg}[1]{\left\langle #1 \right\rangle}
\newcommand{\difFrac}[2]{\frac{\dif #1}{\dif #2}}
\newcommand{\pdfFrac}[2]{\frac{\partial #1}{\partial #2}}
\newcommand{\OFL}{\mathrm{OFL}}
\newcommand{\UFL}{\mathrm{UFL}}
\newcommand{\fl}{\mathrm{fl}}
\newcommand{\op}{\odot}
\newcommand{\Eabs}{E_{\mathrm{abs}}}
\newcommand{\Erel}{E_{\mathrm{rel}}}

\begin{document}

\pagestyle{fancy}
\fancyhead{}
\lhead{乐天, 310101674}
\chead{数据结构与算法项目作业:四则计算器}
\rhead{Dec.14th, 2024}

在该项目作业中,实现了一个能够计算多重括号和四则运算的计算器,支持有限位小数和负数运算,但不支持科学计数法。
实现四则运算器的思路并不复杂,与书上基本一致,代码可以分为以下三个部分:1.将输入的中缀表达式改为后缀表达式 2.计算后缀表达式。3.输出。
我的报告也将会基本按照这三个部分分开介绍。
\section{后缀表达式的计算与识别不合法表达式}
在第一部分,我们总假定已经有了vector<string>形式的后缀表达式,在这个vector中,一个string对应一个完整的数字或运算符,例如:\{“23.5”,“12”,“*”\},在第2部分再介绍如何将中缀表达式转化为这种形式后缀表达式。
如何计算:遍历这个vector,准备一个数据结构(这在这个程序里面用的是自建的一个类,但这个类只是栈的拙劣模仿,当成一个简单的栈即可,因此不特别介绍)number储存读到的数字,如果读到运算符,就从number中弹出最后进入的两个数字,结合这个运算符做运算,再把运算结果放回number,待到最后一个运算符完成运算时,number剩下的那个数字就是计算结果。

关于不合法表达式的识别:我同样将不合法表达式的识别放在了这个部分。事实上,不合法表达式可以被简单归为三种:

1.除数为0 

2.出现了不合法的符号.如\&,

3.符号合法,但数字个数与运算符个数不匹配

第一种情况和第二种情况都可以在“从number中弹出最后进入的两个数字,结合这个运算符做运算”这个部分识别,识别到不可运算的符号、做除法识别到除数为0,就输出ILLEGAL。第三种情况看似复杂,但其实只有一种情况(先不考虑负数),那就是“运算符需要的数字个数比现有数字多”,注意,理论上似乎同样存在“运算符需要的数字个数比现有数字少”的情况,但实际上不可能,这是因为在输入中缀表达式的时候,如果少输入了字符,例如将“4*2*3”输为“42*3”或“4 2*3”,前一种4和2会被理解为整体,即42,虽然未必符合使用者意图,但从表达式看是合法的,后一种空格也会被读进来,会被视为第二种情况。所以,只要考虑“运算符需要的数字个数比现有数字多”的情况,针对这种情况,我们只要在每次读到运算符的时候,看看number剩余的数字个数是否不小于2就好。


\section{中缀表达式到后缀表达式的转换}
这一部分实现了从一个string形式的中缀表达式转换为一个vector<string>的后缀表达式。(先不考虑负数)

由vector<string> changeresult贮存结果。

遍历这个中缀表达式的string,如果读到的是一个数字或者小数点,就把它写入另一个string里(程序里用的名字是numberpart),如果不是(相当于读到了一个符号),在numberpart非空的情况,将这个string整体写入changeresult(这一步相当于得到了一个完整的数字)并清空numberpart,然后通过sign函数给这个符号一个优先级标识,具体规则为:“+”和“-”为1,“*”和“/”为2,“(”为一个很大的数字(该程序中为100),“)”为0,其他符号为3(这里为其他符号统一赋了3,是为了保证这些符号被正常读入,从而在后面的计算中被识别),如果读到的这个符号满足以下三种情况中的至少一种,就将其存入栈operators:1.栈是空的、2.目前栈的最后一个符号是左括号3.优先级比栈最后一个符号大。否则,分两种情况:1.如果它是右括号且上一个是左括号,就将这两个直接删除,2.如果它不满足1,就将operators最后一个符号作为一个string弹出到changeresult,重复上述判定,直到这个符号被删除或者被放入栈operators。
遍历结束后,再做一次将numberpart写入changeresult(最后一个数字)、operators的尽数弹出(这个时候,如果表达式正确,剩下的operators均是前一个优先级比后一个小),就是我们要的后缀表达式。

关于负数:负数的判定,在前一个符号不是“)”的情况下连续读到两个符号,且后一个符号是“-”。
负数的处理,将负数例如-5,按照(0-5)处理(这里的括号是必要的)。

补充:

1.为什么括号不匹配的情况最终能被识别,不论是多了个左括号还是右括号,都会导致左、右括号删不掉,在最后operators的尽数弹出时,就会被写入后缀表达式changeresult,括号同样属于不可运算的符号,就会在计算中被识别,输出ILLEGAL。

2.对于小数点前无数字的输入,同样看成不合法,即.9这种输入,这种错误能识别是因为,计算机不能识别.9为数字,导致计算的时候,“.9”被识别为符号,从而无法进行,相当于和第2种错误用了同样的判别方法。

\section{main函数、测试样例与时间复杂度}
在这个程序早期设计中,报错是由std::cout<<"ILLEGAL"<<std::endl;和exit(EXIT\_FAILURE);这两句话实现,后果是,一旦输入错误的错误的表达式,就直接中止程序了,这个程序的初衷是设计一个计算器,如果十个运算式因为第一个不合法导致后面的都做不了,还要重启程序,作为计算器实在是令人难以接受,为此,从计算器的功能考虑,将计算部分的输出改为了string形式,这样不仅达成了“如果输入的中缀表达式合法,输出计算结果;如果非法,输出ILLEGAL”的要求,而且该计算器可以多次使用。(当然,一件比较尴尬的事情是,为了输出好看一点,在输出结果的函数里,做了一次判定,如果结果不是ILLEGAL,就把string转为double)
 
在最终的main函数里,每次输出用户输入的计算式的计算结果后(不论是不是ILLEGAL)都会提示“按1继续使用,按0退出使用”,用户可以根据自己的需求选择。

以下是一些测试样例和输出结果,保留了部分测试样例在main函数的注释里。

运算次序

5.1+3*23  Result:74.1

45-50/10  Result:40

括号使用 

(5.1-5)*23 Result:2.3

(3.2+1.4*2)*2 Result:12

((5+3)*20/5)   Result:32

3.2*(5.4+4.6) Result:32

负数

8*-2+-7*2 Result:-30

(5+-3)/3 Result:0.666667

四个错误测试

5.1+3*23)  (括号不匹配)输出:ILLEGAL


(5+3)\#*23   (符号不合法)输出:ILLEGAL

((5+3)*23/0) (除数为0)输出:ILLEGAL

(.5-3)*23  (不合法的数字输入)输出:ILLEGAL

该程序(输入一个中缀表达式,得到计算结果)的时间复杂度为$\theta(n)$

\end{document}

%%% Local Variables: 
%%% mode: latex
%%% TeX-master: t
%%% End: 

