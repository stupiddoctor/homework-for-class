\documentclass[UTF8]{ctexart}
\usepackage{geometry, CJKutf8}
\geometry{margin=1.5cm, vmargin={0pt,1cm}}
\setlength{\topmargin}{-1cm}
\setlength{\paperheight}{29.7cm}
\setlength{\textheight}{25.3cm}

% useful packages.
\usepackage{amsfonts}
\usepackage{amsmath}
\usepackage{amssymb}
\usepackage{amsthm}
\usepackage{enumerate}
\usepackage{graphicx}
\usepackage{multicol}
\usepackage{fancyhdr}
\usepackage{layout}
\usepackage{listings}
\usepackage{float, caption}

\lstset{
    basicstyle=\ttfamily, basewidth=0.5em
}

% some common command
\newcommand{\dif}{\mathrm{d}}
\newcommand{\avg}[1]{\left\langle #1 \right\rangle}
\newcommand{\difFrac}[2]{\frac{\dif #1}{\dif #2}}
\newcommand{\pdfFrac}[2]{\frac{\partial #1}{\partial #2}}
\newcommand{\OFL}{\mathrm{OFL}}
\newcommand{\UFL}{\mathrm{UFL}}
\newcommand{\fl}{\mathrm{fl}}
\newcommand{\op}{\odot}
\newcommand{\Eabs}{E_{\mathrm{abs}}}
\newcommand{\Erel}{E_{\mathrm{rel}}}

\begin{document}

\pagestyle{fancy}
\fancyhead{}
\lhead{乐天, 310101674}
\chead{数据结构与算法第七次作业}
\rhead{Nov.31st, 2024}

\section{堆排序的实现}
建了一个名为BinaryHeap的类,该类具有二叉堆的结构,并满足父节点的值小于子节点的值。BinaryHeap支持基本的堆操作,比如插入(insert)与删除最小元(DeleteMin)。该类中私有成员有currentSize(记录当前堆中元素个数,空堆currentSize=0)和一个用于存储堆中元素的vector:array。

insert函数:功能:向堆中插入新元素,并保持堆的性质。

实现细节:将新元素添加到堆的末尾(array[currentSize]),然后通过私有函数percolateUp 函数向上调整,确保父节点小于子节点,最后将currentSize加1。percolateUp函数是通过比较当前节点与其父节点的大小,如果当前节点较小,则通过std::swap与父节点交换,并且如果发生了交换,继续递归进行percolateUp,直到满足堆的性质。


DeleteMin函数:类中有两种DeleteMin函数,一种不带参数,功能是移除最小元,一种带参数,功能是将当前堆中最小元的值传给参数,并将最小元移除

实现细节:(只说明不带参数的DeleteMin函数,带参数的DeleteMin函数只要提前将堆中根节点的值通过引用赋给参数,再运行不带参数的DeleteMin函数即可。)将最右侧的叶子节点赋值给根节点,将currentSize减1,再通过percolateDown函数向下调整。(所以实际上最右侧的叶子节点的值还保留在array中,但由于currentSize已经减1,并不会对降堆操作产生影响,如果之后有新元素insert,还可以覆盖这个值)
percolateDown是通过比较当前节点与其子节点的大小,如果当前节点较大,则将子节点中较小的一个通过std::swap与该节点交换值,继续递归进行percolateDown,直到出现下面情况:1.没有子节点 2.子节点均大于该节点的值。
\\


insert函数与DeleteMin函数均为$\theta(logn)$
\\


接下来通过void myheapsort函数完成堆排序,该函数有一个vector类型的参数,功能是将该vector的元素按从小到大排列。

具体实现:建立空堆,将vector中的所有元素通过insert插入到堆中。然后通过 带参数的DeleteMin 函数依次将堆中的最小元素放回vector中,从而实现排序。

\section{测试程序}
首先生成了长为1000000的随机序列;长为1000000的有序序列;长为1000000的逆序序列;长为1000000的部分元素重复序列各两个,分别使用std::make\_heap、std::sort\_heap排序和myheapsort排序,通过chrono计时,结果如下


\begin{table}[!htp]
    \centering
    \caption{时间比较(单位:毫秒)} 
    \begin{tabular}{c|c|c}
        \hline
         \ & my heapsort time & std::sort\_heap time \\
        \hline
        random sequence & 127 & 120\\
        \hline
        ordered sequence & 65 & 56 \\
        \hline
        reverse sequence & 83 & 61 \\
        \hline
        repetitive sequence & 122 & 117 \\
        \hline
    \end{tabular}
 \end{table}




\section{时间复杂度分析}
由主定理,insert函数与DeleteMin函数的递归满足$O(n)=O(n/2)+\theta(1)$,故均为$\theta(logn)$

在myheapsort函数中进行了n(元素个数)次insert与n次DeleteMin,所以复杂度为$\theta(nlogn)$,推测效率不如std::sort\_heap排序是因为在这个过程中进行了大量的值的复制
\end{document}

%%% Local Variables: 
%%% mode: latex
%%% TeX-master: t
%%% End: 