\documentclass[UTF8]{ctexart}
\usepackage{geometry, CJKutf8}
\geometry{margin=1.5cm, vmargin={0pt,1cm}}
\setlength{\topmargin}{-1cm}
\setlength{\paperheight}{29.7cm}
\setlength{\textheight}{25.3cm}

% useful packages.
\usepackage{amsfonts}
\usepackage{amsmath}
\usepackage{amssymb}
\usepackage{amsthm}
\usepackage{enumerate}
\usepackage{graphicx}
\usepackage{multicol}
\usepackage{fancyhdr}
\usepackage{layout}
\usepackage{listings}
\usepackage{float, caption}

\lstset{
    basicstyle=\ttfamily, basewidth=0.5em
}

% some common command
\newcommand{\dif}{\mathrm{d}}
\newcommand{\avg}[1]{\left\langle #1 \right\rangle}
\newcommand{\difFrac}[2]{\frac{\dif #1}{\dif #2}}
\newcommand{\pdfFrac}[2]{\frac{\partial #1}{\partial #2}}
\newcommand{\OFL}{\mathrm{OFL}}
\newcommand{\UFL}{\mathrm{UFL}}
\newcommand{\fl}{\mathrm{fl}}
\newcommand{\op}{\odot}
\newcommand{\Eabs}{E_{\mathrm{abs}}}
\newcommand{\Erel}{E_{\mathrm{rel}}}

\begin{document}

\pagestyle{fancy}
\fancyhead{}
\lhead{乐天, 310101674}
\chead{数据结构与算法第四次作业}
\rhead{Oct.20th, 2024}

\section{测试程序的设计思路}
测试程序可分为六个部分

在第一部分,主要测试构造函数与赋值运算,第二部分测试重构运算符==、!=、++、---、*的运算,第三部分测试插入功能,第四部分测试修改功能(front、back函数能否正确完成链表的修改),第五部分测试删除功能,第六部分测试清空功能。每个部分涉及到链表打印的部分调用了自行添加的List公有成员函数printall

第一部分,创建了一个名为 intlist1 的整型空链表,一个提供了初始化列表 {1,2,3,4,5}的intlist2整型链表,利用拷贝构造函数创建了和intlist2相同的intlist3,输出intlist3,稍微更改intlist3,再进行连等号的赋值运算,将intlist3值赋给intlist2和intlist1,输出intlist1检验,输出结果应为
\\

intlist3:

1 2 3 4 5 

intlist1:

1 2 3 4 5 1
\\

第二部分,输出结果应为
\\

迭代器相等: 1

迭代器不等: 1

intlist1的第一和第三个数据分别是:1和3

intlist1的倒数第一和第三个数据分别是:1和4
\\

第三部分,检验insert、pushfront、pushback三个函数,最后打印 intlist1 的新状态以及检验对应size大小,输出结果应为
\\

插入功能测试

intlist1:

-8 1 20 22 2 3 4 5 1 12 

size:10
\\

第四部分,利用front与back,观察是否可以顺利更改头尾值,输出应为:
\\

修改头尾测试

100 1 20 22 2 3 4 5 1 200 
\\

第五部分,测试删除功能,分别检验erase的单个删除、pop\_front、pop\_back的功能,这是因为后两个函数使用了erase函数,这样测试更清楚是哪一部分出了问题
再检验erase指定范围删除功能,同时检验size是否正确
输出应为:
\\

删除功能测试

删除第二个元素100 20 22 2 3 4 5 1 200

intlist只保留头尾:100 200 

size:2
\\

第六部分,利用clear清空,利用empty检验是否清空,输出结果应为:
\\

清空功能测试

清空intlist1:

已清空



\section{测试的结果}

测试结果一切正常。
测试结果与描述一致。
最后,我用 valgrind 进行测试,发现没有发生内存泄露。

\section{(可选)bug报告}

我发现了bug,触发条件如下:

\begin{enumerate}
    \item auto it=intlist1.end();
    ++it 
    程序报错
    \item 在erase(iterator from, iterator to)函数里,让前面的迭代器from指向的节点在to的后面 ,程序报错
\end{enumerate}

据我分析,1出现的原因是:设计++、(--)操作没有考虑边界条件,当指针已经指向链表的哨兵元素时,再次使用会超出链表的界限导致报错
2出现的原因,不论是修改前还是修改后的erase,每次循环时都是从from往后找,如果to指向的在from的前面,循环无法正确中止

\end{document}

%%% Local Variables: 
%%% mode: latex
%%% TeX-master: t
%%% End: 

