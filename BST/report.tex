\documentclass[UTF8]{ctexart}
\usepackage{geometry, CJKutf8}
\geometry{margin=1.5cm, vmargin={0pt,1cm}}
\setlength{\topmargin}{-1cm}
\setlength{\paperheight}{29.7cm}
\setlength{\textheight}{25.3cm}

% useful packages.
\usepackage{amsfonts}
\usepackage{amsmath}
\usepackage{amssymb}
\usepackage{amsthm}
\usepackage{enumerate}
\usepackage{graphicx}
\usepackage{multicol}
\usepackage{fancyhdr}
\usepackage{layout}
\usepackage{listings}
\usepackage{float, caption}

\lstset{
    basicstyle=\ttfamily, basewidth=0.5em
}

% some common command
\newcommand{\dif}{\mathrm{d}}
\newcommand{\avg}[1]{\left\langle #1 \right\rangle}
\newcommand{\difFrac}[2]{\frac{\dif #1}{\dif #2}}
\newcommand{\pdfFrac}[2]{\frac{\partial #1}{\partial #2}}
\newcommand{\OFL}{\mathrm{OFL}}
\newcommand{\UFL}{\mathrm{UFL}}
\newcommand{\fl}{\mathrm{fl}}
\newcommand{\op}{\odot}
\newcommand{\Eabs}{E_{\mathrm{abs}}}
\newcommand{\Erel}{E_{\mathrm{rel}}}

\begin{document}

\pagestyle{fancy}
\fancyhead{}
\lhead{乐天, 310101674}
\chead{数据结构与算法第六次作业}
\rhead{Nov.9th, 2024}

\section{对修改后remove函数实现的阐述}
remove函数的逻辑仍然是先查找所要删除的节点位置,再根据其子节点的个数进行处理。

首先,在remove函数的最后一行加上了balance(t)(balance的作用是对指针t指向的节点进行旋转平衡操作并更新相关节点的height),这样当remove进行查找和删除操作的时候,就可以对删除的节点的所有父节点进行height的更新和旋转操作,使其符合AVL树的平衡结构。

对于只有一个子节点的情形,由于逻辑是直接将删除节点的父节点指向其子节点,然后delete掉删除节点,所以其需要balance的节点仅有“删除的节点的所有父节点”,因此上述操作是充分的

对于两个节点的情形,逻辑是这样的:借助detachMin函数找到删除节点(以下简称A节点)的右子树的最小节点(以下简称B节点,用指针\_replace指向),并将B节点从原来的树中分离出来。然后,利用节点替换的方式,实现删除。具体来说,将B节点的左右指针指向A节点的左右指针指向的节点,将A的父节点指向A的指针改为指向B(t=\_replace),最后,用delete删除A节点。

所有,如果只在最后一行加上了balance(t),相当于只balance了t指向的节点及其所有父节点,但由于我们将B节点从原来的树中分离,因此,其实需要进行balance操作的是B节点的所有父节点,所以,对于两个节点的情况,我们先用一个指针to指向B节点的右节点,再进行上述指针移动的操作,最后,我们对t和to指向的两个节点
的访问路径上所有节点height由小到大进行balance操作,这一步由balfromto函数实现(balfromto(t->right,to))

balfromto定义如下

void balfromto(BinaryNode *\&from,BinaryNode *\&to)

    \{
    
    if(from == to)\{
            
        \ \}else\{
        
        \ \ balfromto(from->left,to);
            
        \}
        
        balance(from);

    \}
\\







\end{document}

%%% Local Variables: 
%%% mode: latex
%%% TeX-master: t
%%% End: 

