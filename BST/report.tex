\documentclass[UTF8]{ctexart}
\usepackage{geometry, CJKutf8}
\geometry{margin=1.5cm, vmargin={0pt,1cm}}
\setlength{\topmargin}{-1cm}
\setlength{\paperheight}{29.7cm}
\setlength{\textheight}{25.3cm}

% useful packages.
\usepackage{amsfonts}
\usepackage{amsmath}
\usepackage{amssymb}
\usepackage{amsthm}
\usepackage{enumerate}
\usepackage{graphicx}
\usepackage{multicol}
\usepackage{fancyhdr}
\usepackage{layout}
\usepackage{listings}
\usepackage{float, caption}

\lstset{
    basicstyle=\ttfamily, basewidth=0.5em
}

% some common command
\newcommand{\dif}{\mathrm{d}}
\newcommand{\avg}[1]{\left\langle #1 \right\rangle}
\newcommand{\difFrac}[2]{\frac{\dif #1}{\dif #2}}
\newcommand{\pdfFrac}[2]{\frac{\partial #1}{\partial #2}}
\newcommand{\OFL}{\mathrm{OFL}}
\newcommand{\UFL}{\mathrm{UFL}}
\newcommand{\fl}{\mathrm{fl}}
\newcommand{\op}{\odot}
\newcommand{\Eabs}{E_{\mathrm{abs}}}
\newcommand{\Erel}{E_{\mathrm{rel}}}

\begin{document}

\pagestyle{fancy}
\fancyhead{}
\lhead{乐天, 310101674}
\chead{数据结构与算法第五次作业}
\rhead{Nov.5th, 2024}

\section{对修改后remove函数实现的阐述}
remove函数其余部分并未修改,主要修改的是“被删除的节点有两个子节点”的部分。

在该部分中,首先借助detachMin函数找到删除节点(以下简称A节点)的右子树的最小节点(以下简称B节点),并将B节点从原来的树中分离出来。


然后,利用节点替换的方式,实现删除。具体来说,将B节点的左右指针指向A节点的左右指针指向的节点,将A的parents指向A的指针改为指向B,最后,用delete删除A节点。
\\





\section{测试输出结果的呈现和分析}
保留了原来对该树的其他功能的测试(包括异常处理)

对remove的函数功能进行了测试,分别测试了 对空树进行删除、删除有两个子节点的节点(删除5)、删除叶子节点(删除3)、删除仅含有一个子节点的节点(删除7,注:在执行前面的删除操作后7只有一个节点)、删除不存在于该树中的节点(删除20)、测试的输出结果应为

empty Tree after removing :

Empty tree

Tree after removing 5:\\3\\6\\7\\9\\10\\15\\18

Tree after removing 3:\\6\\7\\9\\10\\15\\18

Tree after removing 7:\\6\\9\\10\\15\\18

Tree after removing 20:\\6\\9\\10\\15\\18

对于删除节点5的情况进行跟踪分析:

根节点为10,比x(5)大,递归调用remove(t->left),
因为 x 等于 t->left->element(都是5),进入 else if 分支。

节点5有两个子节点(3和7),因此调用 detachMin(t->right) 找到右子树中的最小节点,即7的左子树6
。
节点6从原树中分离,6没有右节点,因此7的左指针改为指向空值,指针“\_replace”指向6,然后将6的左指针指向3,右指针指向7,将10的左指针指向6,delete删掉5,实现删除。

\end{document}

%%% Local Variables: 
%%% mode: latex
%%% TeX-master: t
%%% End: 

